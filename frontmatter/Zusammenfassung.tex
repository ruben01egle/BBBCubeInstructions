%%
%%
%%
\thispagestyle{empty}
\chapter*{Kurzfassung}
%
{ \itshape
Muster für eine Kurzfassung, maximal eine Seite }\\


Axiallüfter finden weite Verbreitung, z. B. zur Kühlung von Kraftfahrzeugmotoren. Im Hinblick auf den Energieverbrauch soll der Lüfterwirkungsgrad möglichst hoch sein. Ein wichtiger Parameter zur Optimierung des Wirkungsgrades ist die Drallverteilung längs des Radius.

Die Aufgabe bei der vorliegenden Arbeit bestand darin, eine bezüglich des Wirkungsgrades möglichst günstige Drallverteilung bei typischen Axiallüftern zur Kühlung von Kraftfahrzeugmotoren zu finden. 

Zur Lösung der gestellten Aufgabe wurden nach sorgfältiger Versuchsplanung zwei 
Laufradvarianten nach dem Verfahren von C. Müller ausgelegt, 
in der Versuchswerkstatt gefertigt und auf dem Lüfter-Normprüfstand im Labor XY bei der Firma ABC untersucht:
\bi
	\item Variante A mit konstantem Drall,
	\item Variante B mit nach außen linear zunehmendem Drall.
\ei
Als Ergebnis der Untersuchungen zeigt sich, dass die Schaufeln der Variante A räumlich stark verwunden sind. Variante B besitzt wenig verwundene, leichter zu fertigende Schaufeln. Die Kennlinie der Variante A ist instabil, die der Variante B ist stabil, was sich insbesondere bei Förderung gegen höheren Druck günstig auswirkt. Der Bestwirkungsgrad der Variante B ist mit 55 \% aber schlechter als der der Variante A (60 \%). Die aufgrund der Rechnung zu erwartende Beeinflussung des Nachstroms hinter den Laufrädern ist bei beiden Varianten nicht beobachtbar.

Aus den Untersuchungen ist zu vermuten, dass eine hyperbolische Drallverteilung längs des Radius den Nachteil des schlechteren Bestwirkungsgrades der Variante B nicht aufweist, sonst aber alle ihre Vorteile beibehält. Weitere Untersuchungen zur Bestätigung dieser Vermutung sind nötig.


\cleardoublepage