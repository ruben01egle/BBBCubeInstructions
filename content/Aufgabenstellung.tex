\chapter{Aufgabenstellung}
    Das Zeil des Labors ist es, den Würfel in der instabilen Ruhelage zu stabilisieren. Dazu soll das in der Vorlesung erarbeitete Modell genutzt werden, um einen Zustandsregler zu entwerfen, zu simulieren und auf dem Mikrocontroller umzusetzen. Dazu muss zuerst die Entwicklungsumgebung, wie in \autoref{ch:SetupEntwicklungsumgebung} beschrieben, eingerichtet werden. Anschließend kann die schrittweise Implementierung der Target-Anwendung auf dem BeagleBone durchgeführt werden. Auch gibt es bereits Klassen zum Speichern aller möglichen Daten (Imu-Daten, Zustandsvektor etc.). Verwenden Sie diese und machen Sie sich mit dem Framework vertraut, indem Sie sich anschauen, welche Klassen existieren! Das empfohlene Vorgehen zur Realisierung der Regelung des Würfels ist das Folgende:
    
    \section{Implementierung der Multithread-Umgebung}
    \section{Implementierung des Timings}
    \section{Ansteuerung der Hardware}
    \section{Implementierung der Datenübertragung zum Entwicklungsrechner}
    \section{Kalibrierung der Sensordaten}
    \section{Implementierung der Zustandsschätzung und Filterung}
    \section{Reglerauslegung und Simulation mit Simulink}
    \section{Implementierung der Regelschleife}
    